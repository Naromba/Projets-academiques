\documentclass{article}

\usepackage[french]{babel}
\usepackage[utf8]{inputenc}
\usepackage[T1]{fontenc}
\usepackage{amsfonts}            %For \leadsto
\usepackage{amsmath}             %For \text
\usepackage{fancybox}            %For \ovalbox
\usepackage{hyperref}
\usepackage{color}

\DeclareUnicodeCharacter{03BB}{\ensuremath{{\color{black}{\lambda}}}}

\title{Travail pratique \#2}
\author{IFT-2035}

\begin{document}

\maketitle

\newcommand \mML {\ensuremath\mu\textsl{ML}}
\newcommand \kw [1] {\textsf{#1}}
\newcommand \id [1] {\textsl{#1}}
\newcommand \punc [1] {\kw{`#1'}}
\newcommand \str [1] {\texttt{"#1"}}
\newenvironment{outitemize}{
  \begin{itemize}
  \let \origitem \item \def \item {\origitem[]\hspace{-18pt}}
}{
  \end{itemize}
}
\newcommand \MAlign [2][t] {\begin{array}[#1]{@{}l} #2 \end{array}}

\section{Survol}

Ce TP vise à améliorer la compréhension des types et de la portée statique
en modifiant le code du travail précédent.  Les étapes de ce
travail sont les suivantes:
\begin{enumerate}
\item Lire et comprendre cette donnée.  Cela prendra probablement une partie
  importante du temps total.
\item Lire, trouver, et comprendre les parties importantes du code fourni.
\item Compléter et ajuster le code fourni.
\item Écrire un rapport.  Il doit décrire \textbf{votre} expérience pendant
  les points précédents: problèmes rencontrés, surprises, choix que vous
  avez dû faire, options que vous avez sciemment rejetées, etc...  Le
  rapport ne doit pas excéder 5 pages.
\end{enumerate}

Ce travail est à faire en groupes de deux.  Le rapport, au format
\LaTeX\ exclusivement (compilable sur \texttt{ens.iro}) et le code sont
à remettre par remise électronique avant la date indiquée.  Aucun retard ne
sera accepté.  Indiquez clairement votre nom au début de chaque fichier.

Ceux qui veulent faire ce travail seul(e)s doivent d'abord en obtenir
l'autorisation, et l'évaluation de leur travail n'en tiendra pas compte.
Des groupes de 3 ou plus sont \textbf{exclus}.

\newpage
\section{SSlip: Un Slip typé ssstatiquement}

\begin{figure}
  \begin{displaymath}
    \begin{array}{r@{~}r@{~}l@{~~}l}
      \tau &::=& \kw{Num} & \text{Type des nombres entiers} \\
      &\mid& \kw{Bool} & \text{Type des booléens} \\
      &\mid& (\tau_1~...~\tau_n \to \tau) & \text{Type d'une fonction}
       \medskip \\
      e &::=& n & \text{Un entier signé en décimal} \\
      &\mid& x & \text{Une variable}  \\
      &\mid& (:~e~\tau) & \text{Annotation de type}  \\
      &\mid& (\kw{if}~e~e_{\kw{then}}~e_{\kw{else}}) &
           \text{Expression conditionelle} \\
      &\mid& (e_0~e_1~...~e_n) &
           \text{Un appel de fonction} \\
      &\mid& (\kw{fob}~((x_1~\tau_1)~...~(x_n~\tau_n))~e) & \text{Une fonction} \\
      &\mid& (\kw{let}~x~e_1~e_2) &
           \text{Déclaration locale simple} \\
      &\mid& (\kw{fix}~(d_1~...~d_n)~e) &
           \text{Déclarations locales récursives} \\
      &\mid& + \mid - \mid * \mid / \mid ... &
           \text{Opérations prédéfinies}
       \medskip \\
      d &::=& (x~e) & %% \mid ((x~x_1~...~x_n)~e)
          \text{Déclaration de variable} \\
      &\mid& (x~\tau~e) & %% \mid ((x~x_1~...~x_n)~e)
          \text{Déclaration typée} \\
      &\mid& ((x~(x_1~\tau_2)~...~(x_n~\tau_n))~e) &
           \text{Déclaration de fonction} \\
      &\mid& ((x~(x_1~\tau_2)~...~(x_n~\tau_n))~\tau~e) &
           \text{Déclaration complète}
    \end{array}
  \end{displaymath}
  \caption{Syntaxe de SSlip}
  \label{fig:syntaxe}
\end{figure}

\noindent
Vous allez travailler sur l'implantation d'une version de Slip avec typage
statique.  La syntaxe de ce langage est décrite à la
Figure~\ref{fig:syntaxe}.  Par rapport à Slip, les changements sont:
\begin{itemize}
\item l'ajout de la forme $(:~e~\tau)$ qui se comporte comme $e$ mais qui de
  plus annonce que cette expression doit avoir le type $\tau$, ce qui sera
  vérifié pendant la vérification des types.
\item L'ajout de type qui accompagne dorénavant chaque argument de fonction.
\item La possibilité d'ajouter une annotation de type aux déclarations.
\end{itemize}
La sémantique dynamique de SSlip est la même que celle de Slip, modulo le
fait qu'il y a maintenant des annotations de types qu'il faut ignorer.

\subsection{Sucre syntaxique}

\noindent
La première partie du travail est la même que celle du premier TP, qui est
de transformer le code du format \id{Sexp} à \id{Lexp}.  Vous pourrez en
grande partie reprendre votre code (ou celui du solutionnaire) pour ça, mais
il faudra l'ajuster à la nouvelle syntaxe et au nouveau sucre syntaxique.
Le sucre syntaxique de SSlip est similaire à celui de Slip, mais il
y a maintenant plus de cas:
\begin{displaymath}
  \begin{array}{r@{\;\;\;\;\Longleftrightarrow\;\;\;\;}l}
    (x~\tau~e) & (x~(:~e~\tau)) \\
    ((x~\textsl{args...})~e) &
        (x~(\kw{fob}~(\textsl{args...})~e)) \\
    ((x~\textsl{args...})~\tau~e) &
        (x~(\kw{fob}~(\textsl{args...})~(:~e~\tau)))
  \end{array}
\end{displaymath}

\subsection{Sémantique statique}

\newcommand \CheckKind [1] {\vdash #1}
\newcommand \CheckType [2][\Gamma] {#1 \vdash #2 ~:~}
\newcommand \InferType [2][\Gamma] {#1 \vdash #2 ~:~}
\newcommand \BothType [2][\Gamma] {#1 \vdash #2 ~:~}
\newcommand \Axiom [2] { \mbox{}\hspace{10pt}\nolinebreak\ensuremath{\displaystyle
    \frac{#1}{#2}}\nolinebreak\hspace{10pt}\mbox{} \medskip}
\newcommand \HA {\hspace{17pt}}
\renewcommand \: {\!:\!}

\begin{figure}
  \begin{minipage}{\columnwidth}
    \noindent \centering
    \Axiom{}{\InferType{n}{\kw{Num}}}
    \Axiom{\Gamma(x) = \tau}{\InferType{x}{\tau}}
    \Axiom{\CheckType{e}{\tau}}
          {\InferType{(\!\kw{:}~e~\tau)}{\tau}}
    %% \Axiom{\InferType{e}{\tau}}
    %%       {\CheckType{e}{\tau}}
    \Axiom{\CheckType{e_1}{\kw{Bool}}
           \HA \CheckType{e_2}{\tau}
           \HA \CheckType{e_3}{\tau}}
          {\CheckType{(\kw{if}~e_1~e_2~e_3)}{\tau}}
    \Axiom{\InferType{e_0}{(\tau_1~...~\tau_n\to\tau_r)}
           \HA
           \forall i.\;\; \CheckType{e_i}{\tau_i}}
          {\InferType{(e_0~e_1~...~e_n)}{\tau_r}}
    \Axiom{\CheckType[\Gamma,x_1\:\tau_1,...,x_n\:\tau_n]{e}{\tau_r}}
          {\CheckType{(\kw{fob}~((x_1~\tau_1)~...~(x_n~\tau_n))~e)}{(\tau_1~...~\tau_n\to\tau_r)}}
    \Axiom{
      \InferType{e_1} {\tau_1}
      \HA
      \InferType[\Gamma,x\:\tau_1]{e_2} {\tau_2}
    }{
      \InferType{(\kw{let}~x~e_1~e_2)} {\tau_2}
    }
    \Axiom{
      \Gamma' = \Gamma,x_1\:\tau_1,...,x_n\:\tau_n \HA
      \InferType[\Gamma']{e}{\tau}
      \HA
      \forall i. \;\; \InferType[\Gamma']{e_i}{\tau_i}
    } {
      \InferType{(\kw{fix}~((x_1~e_1)~...~(x_n~e_n))~e)} {\tau}
    }
    \mbox{}
  \end{minipage}
  \caption{Règles de typage de SSlip}
  \label{fig:typing}
\end{figure}

Vu que SSlip est typé statiquement, il est accompagné de règles de typage,
qui sont présentées à la Figure~\ref{fig:typing}.  Le jugement de typage
s'écrit $\CheckType{e}{\tau}$ et se lit ``$e$ a type $\tau$ dans l'environnement
$\Gamma$''.  $\Gamma$ contient le type de toutes les variables dont la portée couvre
$e$, i.e.~il contient le type de toutes les variables auxquelles $e$ a le
droit de faire référence.

La deuxième partie du travail est d'implanter la vérification de types, donc
de transformer ces règles en un morceau de code Haskell.  Un détail
important pour cela est que le but fondamental de la vérification de types
n'est pas de trouver le type d'une expression mais plutôt de trouver
d'éventuelles erreurs de typage, donc il est important de
\emph{tout} vérifier.

La vérification de type est implantée par la fonction \id{check}:
\begin{verbatim}
  check :: Bool -> TEnv -> Lexp -> Type
\end{verbatim}
Cette fonction doit être totale: en cas d'erreur elle doit renvoyer un
``type'' de la forme ``\id{Terror}~$s$'' où $s$ est une chaîne de caractères
qui décrit l'erreur.  Le premier argument, un booléen, prend normalement la
valeur \kw{True} et indique que la fonction doit bien vérifier que
l'expression est typée correctement et renvoyer une erreur si ce n'est pas
le cas.  Par contre si l'argument est \kw{False}, cela signifie que
\id{check} peut présumer que l'expression est typée correctement et doit
simplement trouver son type.

Cet argument booléen est utilisé pour \kw{fix}.  Parmis les règles
ci-dessus, la règle du \kw{fix} nécessite de ``deviner'' le type des $x_i$
pour construire $\Gamma'$.  Par exemple avec un code source comme:
\begin{displaymath}
  (\MAlign{
    \kw{fix}~((
    \MAlign{(\id{fact}~(n~\kw{Num})) \\
            ~~(\kw{if}~(>~n~0)~(*~n~(\id{fact}~(-~n~1)))~1)))} \\
    ~~...)}
\end{displaymath}
Pour pouvoir vérifier les types dans le corps de la fonction, il faut
connaître le type de \id{fact} dans l'environnement.  Mais les
annotations de types ne nous donnent que le type de son argument et non
celui de sa valeur de retour.  Cependant, on peut trouver le type de la
valeur de retour de \id{fact} en consultant son code.

Donc chaque expression $e_i$ de \kw{fix} sera d'abord passée à ``\id{check}
\kw{False}'' (avec un environnement qui donne un type invalide aux nouvelles
variables, vu qu'on ne connaît pas encore leur type), pour essayer de
trouver justement le type de chaque $e_i$, de manière à ``deviner'' le $\Gamma'$
qu'on utilise ensuite pour appeler \id{check} une deuxième fois pour chaque
$e_i$, cette fois-ci avec l'argument \kw{True}.

\section{Optimisation}

La troisième partie du travail consiste à implanter une phase d'optimisation
du code, qui transforme des expressions de type \id{Lexp} en \id{Dexp}.
Cette optimisation fait deux changements:
\begin{itemize}
\item Effacement des types: après avoir vérifier les types, on a plus besoin
  de cette information pour exécuter le code, donc on peut effacer toutes
  les informations relatives aux types.
\item De même la loi du renommage-$\alpha$ nous garanti que les noms de variables
  n'affectent pas l'exécution, et on s'en débarrasse aussi, en les
  remplaçant par des informations de position dans l'environnement, qu'on
  appelle des \emph{index de De Bruijn}, du nom du mathématicien qui les
  a inventés.
\end{itemize}

\subsection{Indexes de De Bruijn}

En portée statique, l'environnement d'évaluation à un point du programme
a toujours la même forme, avec les même variables placées dans le même ordre
(d'ailleurs aussi les mêmes variables, dans le même ordre que dans le $\Gamma$
utilisé lors de la vérification des types), et seules les valeurs de ces
variables peu changer.  Donc au lieu de rechercher la variable par son nom
à chaque fois, on peut remplacer le nom de la variable par sa position dans
l'environnement, ce qui permet un accès beaucoup plus rapide où il n'est
plus nécessaire de comparer des noms de variables.

l'utilisation des indexes de De Bruijn élimine les noms de variables dans
les déclarations et remplace les références par des nombres qui indiquent la
position de la variable à laquelle on fait référence.  Concrètement, si on
considère un $\lambda$-calcul simple, la syntaxe habituelle:
\begin{displaymath}
  e ::=  c ~|~ x ~|~ \lambda x \to e ~|~ e_1~e_2
\end{displaymath}
devient
\begin{displaymath}
  e ::=  c ~|~ \#n ~|~ \lambda \to e ~|~ e_1~e_2
\end{displaymath}
où $\lambda \to e$ est une fonction qui prend un argument anonyme, et $\#n$ est une
référence à le $n$-ième variable de l'environnement (0 étant la variable la
plus récente/proche).
Une expression du $\lambda$-calcul comme:
\begin{displaymath}
  \lambda x \to \lambda y \to \kw{let}~z = x + y~\kw{in}~z + x
\end{displaymath}
devient:
\begin{displaymath}
  \lambda \to \lambda \to \kw{let}~\#1 + \#0~\kw{in}~\#0 + \#2
\end{displaymath}
On voit ici que les $\lambda$ ne portent plus le nom de leur argument (vu qu'il
est maintenant anonyme), et de même pour le nom de la variable introduite
par le \kw{let}.  La référence à $y$ a été remplacée par $\#0$ parce que,
à ce point du code, l'environnement contient (par ordre de proximité) $y,
x$ et donc $y$ est en position 0.  La référence à $z$ a aussi été remplacée
par $\#0$ vu qu'à ce point du code l'environnement contient $z, y, x$ et
donc $z$ est aussi en position 0.  La première référence à $x$ a été
remplacée par $\#1$ vu qu'à cet endroit l'environnement contient $y,x$,
alors que la deuxième est remplacée par $\#2$ vu qu'à cet endroit
l'environnement contient $z,y,x$.

\section{Travail}

La donnée est similaire à celle du TP1 modifiée légèrement pour
y inclure le squelette du TP2 (et changer un peu le format de sortie de
\kw{run}), donc ce qu'il vous reste à faire est:
\begin{itemize}
\item \kw{s2l} pour accepter les nouveaux éléments de syntaxe.
\item \kw{check} pour implémenter la vérification des types.
\item \kw{l2d} pour convertir le code aux indexes de De Bruijn.
\item \kw{eval} pour exécuter le code.
\end{itemize}
Pour \kw{s2l} et \kw{eval} je vous recommande de vous inspirer du
solutionnaire du TP1.

Vous devez aussi fournir un fichier \texttt{tests.sslip}, similaire au
\texttt{tests.slip} du TP1 sauf qu'il doit contenir au moins 10 tests, dont
au moins 5 qui sont typés correctement et au moins 5 qui sont \emph{refusés}
par le vérificateur de type, avec une explication de l'erreur de type pour
chacun des tests, Au moins 2 des tests dont le type est incorrect doivent
contenir du code dont l'exécution par \kw{eval} n'échouerait \emph{pas}.

\subsection{Remise}

Pour la remise, vous devez remettre trois fichiers (\texttt{sslip.hs},
\texttt{tests.sslip}, et \texttt{rapport.tex}) par la
page Moodle (aussi nommé StudiUM) du cours.  Assurez-vous que le rapport
compile correctement sur \texttt{ens.iro} (auquel vous pouvez vous connecter
par SSH).

\section{Détails}

\begin{itemize}
\item La note sera divisée comme suit: 25\% pour le rapport, 60\% pour le
  code, et 15\% pour les tests.
\item Tout usage de matériel (code ou texte) emprunté à quelqu'un d'autre
  (trouvé sur le web, ...) doit être dûment mentionné, sans quoi cela sera
  considéré comme du plagiat.
\item Le code ne doit en aucun cas dépasser 80 colonnes.
\item Vérifiez la page web du cours, pour d'éventuels errata, et d'autres
  indications supplémentaires.
\item La note est basée d'une part sur des tests automatiques, d'autre part
  sur la lecture du code, ainsi que sur le rapport.  Le critère le plus
  important, et que votre code doit se comporter de manière correcte.
  Ensuite, vient la qualité du code: plus c'est simple, mieux c'est.
  S'il y a beaucoup de commentaires, c'est généralement un symptôme que le
  code n'est pas clair; mais bien sûr, sans commentaires le code (même
  simple) est souvent incompréhensible.  L'efficacité de votre code est sans
  importance, sauf si votre code utilise un algorithme vraiment
  particulièrement inefficace.
\end{itemize}

\end{document}
